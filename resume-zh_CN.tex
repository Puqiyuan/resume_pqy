% !TEX TS-program = xelatex
% !TEX encoding = UTF-8 Unicode
% !Mode:: "TeX:UTF-8"

\documentclass{resume}
\usepackage{zh_CN-Adobefonts_external} % Simplified Chinese Support using external fonts (./fonts/zh_CN-Adobe/)
% \usepackage{NotoSansSC_external}
% \usepackage{NotoSerifCJKsc_external}
% \usepackage{zh_CN-Adobefonts_internal} % Simplified Chinese Support using system fonts
\usepackage{linespacing_fix} % disable extra space before next section
\usepackage{cite}

\begin{document}
\pagenumbering{gobble} % suppress displaying page number

\name{蒲启元}

\basicInfo{
  \email{pqy7172@gmail.com} \textperiodcentered\ 
  \phone{(+86) 183-145-55392}
  }
 
\section{\faGraduationCap\  教育背景}
\datedsubsection{\textbf{西南林业大学}, 昆明, 云南}{2014 -- 2018}
\textit{学士}\ 计算机科学与技术

\section{\faUsers\ 工作经历}
\datedsubsection{\textbf{绿盟科技} 成都}{2018年6月 -- 2020年3月}
\role{内核开发工程师}{}
\begin{onehalfspacing}
  负责DPI深度包探测框架移植到国产化平台,比如申威、飞腾以及龙芯。负责处理Linux内核问题。
  \begin{itemize}
  \item 编写汇编,将DPI软件移植到申威、飞腾以及龙芯平台。
  \item 处理内核宕机问题。
  \item 移植安全部门的Python代码到C代码并优化提升性能。
  \end{itemize}
\end{onehalfspacing}

\datedsubsection{\textbf{统信软件} 成都}{2020年3月 -- 2021年3月}
\role{内核开发工程师}{}
\begin{onehalfspacing}
  负责处理Linux内核问题。移植Golang编译器到申威、LoongArch平台。
  \begin{itemize}
  \item 处理通过LTP测试集测试内核时暴露出来的各种问题,涉及分析内核里各个模块。
  \item 与申威、龙芯合作移植Golang编译器。
  \end{itemize}
\end{onehalfspacing}

\datedsubsection{\textbf{龙芯中科} 北京}{2021年3月 -- 至今}
\role{内核开发工程师}{}
\begin{onehalfspacing}
负责LoongArch服务器内核性能、稳定问题解决。内核新feature开发。固件ACPI表调试等。
\begin{itemize}
  \item 修复LTP,LTPstress等各种内核测试集测试内核时暴露出来的问题。
  \item 修复mm模块pte flag操作不当引起的内核脏页丢失问题。
  \item 重构LoongArch平台上pte flag属性操作接口。
  \item 分析各种io scheduler对桌面系统交互性的影响。
  \item 排查桥片硬件问题导致nvme驱动报告io timeout。
  \item 修复pci scan扫描所有空间时引起的死机问题。
  \item 优化UnixBench测试在LoongArch服务器平台的性能,编写300+行内核公共代码。
  \item 分析内核网络协议栈,解决客户网络延迟问题。
  \item 优化file操作的性能,对标鲲鹏920。
  \item 编写内核测试自动化bash脚本,提高效率。
  \item 调试ACPI表操作硬件。
  \item 实现可信安全功能。
\end{itemize}
\end{onehalfspacing}

% Reference Test
%\datedsubsection{\textbf{Paper Title\cite{zaharia2012resilient}}}{May. 2015}
%An xxx optimized for xxx\cite{verma2015large}
%\begin{itemize}
%  \item main contribution
%\end{itemize}

%\section{\faCogs\ IT 技能}
% increase linespacing [parsep=0.5ex]
%\begin{itemize}[parsep=0.5ex]
%  \item 编程语言: C == Python > C++ > Java
%  \item 平台: Linux
%  \item 开发: xxx
%\end{itemize}

%\section{\faHeartO\ 获奖情况}
%\datedline{\textit{第一名}, xxx 比赛}{2013 年6 月}
%\datedline{其他奖项}{2015}

\section{\faInfo\ 其它}
% increase linespacing [parsep=0.5ex]
\begin{itemize}[parsep=0.5ex]
  \item 2014-2015年度校级三好学生。
  \item 2015-2016年度省级三好学生。
  \item 优秀毕业生。
  \item 优秀毕业论文:一个简单操作系统的实现。
  \item 2018年云南省大学生计算机作品赛二等奖。
  \item 英文阅读流畅,CET6级。
\end{itemize}

%% Reference
%\newpage
%\bibliographystyle{IEEETran}
%\bibliography{mycite}
\end{document}
